\documentclass[../probability-notes.tex]{subfiles}
\begin{document}
    %%%%%%%%%%%%%%%%%%%%%%%%%%%%%%%%%%%%%%%%%%%%%%%%%%%%%%%%%%%%%%%%%%%%%%%%%%%
    \section{Chi-Square Distribution}\label{chi_square}
    If $Z_{1}, Z_{2}, \ldots, Z_{n}$ are $n$ independent standard normal variables, then the random variable $X$
    \begin{align*}
        X &= Z_{1}^{2} + Z_{2}^{2} + \cdots + Z_{n}^{2}\\
        \text{then, \;} X &\sim \chi_{n}^{2}
    \end{align*}
    i.e., $X$ follows the chi-square distribution with $n$ degrees of freedom.\newline

    If we add two chi-square distributed variables with degrees of freedom $n_{1}$ and $n_{2}$, then the resultant variable itself is chi-square distributed with $n_{1} + n_{2}$ degrees of freedom. This simply follows from the fact that the sum of the two random variables is nothing but sum of $n_{1} + n_{2}$ standard normal squared variables which is nothing but a chi-square variable with $n_{1} + n_{2}$ degrees of freedom.\newline

    If $X \sim \chi_{n}^{2}$, then $\chi_{\alpha, n}^{2}$ is
    \begin{align*}
        P(X \geq \chi_{\alpha, n}^{2}) = \alpha
    \end{align*}
    This quantity is usually listed in mathematical tables since they are heavily used in hypothesis testing.\newline

    %%%%%%%%%%%%%%%%%%%%%%%%%%%%%%%%%%%%%%%%%%%%%%%%%%%%%%%%%%%%%%%%%%%%%%%%%%%
    \subsection{Relation between Chi-Square and Gamma Distribution}\label{sec:rel_gamma_chi}
    Consider the moment generating function for a chi-square random variable with $n=1$ degrees of freedom
    \begin{align*}
        E[e^{tX}] &= E[e^{tZ^{2}}] \text{\; $Z \sim \mathcal{N}(0, 1)$}\\
        &= \int_{-\infty}^{\infty} e^{tx^{2}} f_{Z}(x) dx \text{\; since $E[g(x)] = \int_{x} g(x)p(x)$}\\
        &= \frac{1}{\sqrt{2\pi}} \int_{-\infty}^{\infty} e^{tx^{2}} e^{-x^{2}/2}\\
        &= \frac{1}{\sqrt{2\pi}} \int_{-\infty}^{\infty} e^{-x^{2}(1/2 - t)}\\
        \text{Using \;} \int_{-\infty}^{\infty} e^{-a(x+b)^{2}} &= \sqrt{\frac{\pi}{a}}\\
        E[e^{tX}] &= \frac{1}{\sqrt{2\pi}} \sqrt{\frac{\pi}{1/2 - t}}\\
        &= \frac{1}{\sqrt{1 - 2t}}
    \end{align*}

    Extending this idea to the case of $n$ degrees of freedom,
    \begin{align*}
        E[e^{tX}] &= E[e^{t(Z_{1}^{2} + Z_{2}^{2} + \cdots + Z_{n}^{2})}]\\
        &= E[\prod_{i=1}^{n} e^{t Z_{i}^{2}}]\\
        &= \prod_{i=1}^{n} E[e^{t Z_{i}^{2}}] \text{\; since $Z_{i}$ are independent}\\
        &= (1 - 2t)^{-n/2} \text{\; from the derivation above}
    \end{align*}

    But, the quantity just derived is nothing but the moment generating function of the Gamma distribution with parameters $(n/2, 1/2)$. Hence, by the uniqueness of the moment generating function, we are forced to conclude that the \textbf{probability density function of a chi-square variable with n degrees is same as that of a Gamma distribution with parameters (n/2, 1/2)}.\newline
    Thus,
    \begin{align*}
        f_{X}(x) = \frac{\frac{1}{2} e^{-x/2} (\frac{x}{2})^{(n/2) - 1}}{\Gamma(\frac{n}{2})} \text{\; $x > 0$}
    \end{align*}

    \subsubsection{Sum of Exponentially Distributed Random Variables to Chi-Square Distribution}
    We say that a Gamma distributed random variable with $\lambda = 1/2$ and $\alpha$ can be considered equivalent to a $\chi^{2}_{2\alpha}$ variable. Here, $\lambda$ is contrained to be $1/2$. By transforming the variables appropriately, we can extend the idea to a sum of exponentially distributed random variables.\newline

    Consider $n$ independent and identically exponentially distributed random variables $X_{i}$ with parameter $\lambda$. Consider for any of those random variables $X_{i}$,
    \begin{align*}
        Y &= 2\lambda X\\
        F_{Y}(y) &= P(Y \leq y) = P(X <\leq \frac{y}{2\lambda})\\
        &= F_{X}(\frac{y}{2\lambda}) = 1 - \exp \bigg( \frac{y}{2} \bigg)\\
        f_{Y}(y) &= \frac{d}{dy} 1 - \exp \bigg( -\frac{y}{2} \bigg)\\
        &= \frac{1}{2}\exp \bigg( \frac{y}{2} \bigg)\\
        &= Exp(\frac{1}{2})
    \end{align*}
    i.e., $\frac{2}{\lambda}Exp(\lambda) \sim Exp(\frac{1}{2})$. Now, consider the sum of these transformed random variables
    \begin{align*}
        \frac{2}{\lambda}\bigg( X_{1} + \cdots X_{n} \bigg) &\sim Gamma(n, 1/2) \sim \chi_{2n}^{2}
    \end{align*}

    Hence, we can convert the sum of $n$ exponentially distributed random variables with parameter $\lambda$, to a $\chi^{2}_{n}$ variable by multiplying the individual variables by $2/\lambda$.

    %%%%%%%%%%%%%%%%%%%%%%%%%%%%%%%%%%%%%%%%%%%%%%%%%%%%%%%%%%%%%%%%%%%%%%%%%%%
    \subsection{Mean and Variance}
    Since the distribution of a chi-square variable is identical to a $Gamma(n/2, 1/2)$ distribution,
    \begin{align*}
        \Aboxed{E[X] &= n}\\
        \Aboxed{Var(x) &= 2n}
    \end{align*}

\end{document}
