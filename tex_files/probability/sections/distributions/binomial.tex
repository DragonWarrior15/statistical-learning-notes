\documentclass[../../probability-notes.tex]{subfiles}
\begin{document}
    \subsection{Binomial Random Variable}
    \emph{Binomial Random Variable} $X$ is defined as the number of successes in an experiment with $n$ independent trials, where each trial can only have two outcomes, \emph{success} or \emph{failure}.\\
    Let $X_{i}$ denote the Random Variable corresponding to the individual trials, with probability of success $p$. Then we have the following

    \begin{alignat*}{2}
        X_{i} &= \begin{cases} 1 &\mbox{if success in trial i}\\ 
                                0 &\mbox{otherwise} \end{cases} \tag*{indicator variable} \\
        X &= X_{1} + X_{2} + \cdots + X_{n} = \sum_{i=1}^{n} X_{i} \\
        P(X=k) &= \sum_{k=0}^{n} \binom{n}{k} p^{k} (1 - p)^{n-k}
    \end{alignat*}

    \subsubsection{Mean and Variance}
    First let's calculate the mean and variance for a single trial $X_{i}$
    \begin{alignat*}{2}
        E[X_{i}] &= 1 * p + 0 * (1 - p) &&= p\\
        Var(X_{i}) &= (1 - p)^{2}p + (0-p)^{2}(1-p) &&= p(1-p)
    \end{alignat*}
    
    We know that all $X_{i}'s$ are independent. Hence, the mean and variance for X become
    \begin{alignat*}{3}
        E[X] &= E[\sum_{i=1}^{n} X_{i}] &&= \sum_{i=1}^{n}E[X_{i}] &&= np \\
        Var(X) &= Var(\sum_{i=1}^{n} X_{i}) &&= \sum_{i=1}^{n} Var(X_{i}) &&= np(1-p)
    \end{alignat*}
\end{document}