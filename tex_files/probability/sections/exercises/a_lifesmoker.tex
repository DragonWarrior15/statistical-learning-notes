\documentclass[../../probability-notes.tex]{subfiles}
\begin{document}
    Based on the definitions in section \ref{sec:life_testing}, death rate is simply the hazard function, i.e., $\lambda_{s} = 2 \lambda_{n}$ or death rate in smokers is twice that in non smokers. Now,
    \begin{alignat*}{2}
        P(t > B | t > A) &= \frac{P(t>B, t>A)}{P(t>A)} &&= \frac{P(t>B)}{P(t>A)}\\
        &= \frac{1 - F(B)}{1 - F(A)} &&= \frac{exp(-\int_{0}^{B} \lambda(t) dt)}{exp(-\int_{0}^{A} \lambda(t) dt)}\\
        &= exp(-\int_{A}^{B} \lambda(t) dt)\\
        P_{s}(t > B | t > A) &= exp(-\int_{A}^{B} \lambda_{s}(t) dt) &&= exp(-\int_{A}^{B} 2\lambda_{n}(t) dt)\\
        &= exp(-\int_{A}^{B} \lambda_{n}(t) dt)^{2} &&= P_{n}(t > B | t > A)^{2}
    \end{alignat*}
    or, the conditional probability of survival till an age for a smoker is sqaure that of a non smoker (note that probability $< 1$).
\end{document}