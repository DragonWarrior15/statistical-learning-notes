\documentclass[../../probability-notes.tex]{subfiles}
\begin{document}
        A very straightforward way is to use a triple integral
        \begin{align*}
            P(X < Y < Z) = \int_{0}^{\inf} \int_{0}^{z} \int_{0}^{y} \lambda e^{-\lambda x} \mu e^{-\mu y} \nu e^{-\nu z} dx dy dz = \frac{\lambda \mu}{(\lambda + \mu + \nu)(\mu + \nu)}
        \end{align*}
        $P(X < Y < Z)$ can be broken down as $P(X < min(Y,Z)) P(Y < Z)$.\newline
        Consider just $P(Y < Z)$
        \begin{align*}
            P(Y < Z) = \int_{0}^{\inf} \int_{0}^{z} \mu e^{-\mu y} \nu e^{-\nu z} dy dz = \frac{\mu}{\mu + \nu}
        \end{align*}
        Thus, when two exponential processes are considered, probaility of arrival of 1st before 2nd is simply the percentage ratio of parameters. Thus,
        \begin{align*}
            P(X < min(Y,Z)) &= \frac{\lambda}{\lambda + (\mu + \nu)} \tag*{$Y$ and $Z$ can be combined as a single process}\\
            P(Y < Z) &= \frac{\mu}{\mu + \nu}\\
            P(X < Y < Z) &= P(X < min(Y,Z)) P(Y < Z)\\
                        &= \frac{\lambda \mu}{(\lambda + \mu + \nu)(\mu + \nu)}
        \end{align*}
\end{document}