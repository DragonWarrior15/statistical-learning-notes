\documentclass[../../probability-notes.tex]{subfiles}
\begin{document}
    The most powerful test will follow the Neyman-Pearson Lemma (see section \ref{neyman-pearson}). To apply that, we need to first calculate the likelihood ratio of the five observed values under the two hypothesis.
    \begin{align*}
        L &= \frac{f(X_{1}, \ldots, X_{5} \lvert H_{1})}{f(X_{1}, \ldots, X_{5} \lvert H_{0})} =\frac{\bigg( \frac{1}{\sqrt{2 \pi }} \bigg)^{5} \exp \bigg(-\frac{1}{2} \sum_{i=1}^{5} (X_{i} - 2)^{2} \bigg)}{\bigg( \frac{1}{\sqrt{8 \pi}} \bigg)^{5} \exp \bigg( -\frac{1}{8} \sum_{i=1}^{5} (X_{i} - 2)^{2}\bigg)}\\
        &= 32 \exp \bigg(-\frac{3}{8} \sum_{i=1}^{5} (X_{i} - 2)^{2} \bigg) > \eta\\
        \implies \sum_{i=1}^{5} (X_{i} - 2)^{2} &< -\frac{8}{3} \ln \big( \frac{\eta}{32} \big) = c
    \end{align*}
    Hence, the critical region (rejecting $H_{0}$) for the test is defined as
    \begin{align*}
         \sum_{i=1}^{5} (X_{i} - 2)^{2} < c
     \end{align*}

     Note that the above sum is a scaled version of the $\chi_{5}^{2}$ variable. Using the size of test, $\alpha = 0.05$,
     \begin{align*}
         P(\text{Reject}\; H_{0} \lvert H_{0} \; \text{is true}) &= P \bigg(\sum_{i=1}^{5} (X_{i} - 2)^{2} < c \lvert \sigma^{2} = 4 \bigg)\\
         \implies P \bigg(\sum_{i=1}^{5} \bigg(\frac{X_{i} - 2}{2} \bigg)^{2} < \frac{c}{4} \bigg) &= 0.05\\
         P \bigg(\chi_{5}^{2} \geq \frac{c}{4} \bigg) &= 1 - 0.05 = 0.95\\
         \implies \frac{c}{4} &= \chi_{5, 0.95}^{2} = 1.15\\
         c &= 4.6
     \end{align*}

     Hence, the critical region is given by
     \begin{align*}
         C = \bigg\{ (X_{1}, \ldots, X_{5}) : \sum_{i=1}^{5} (X_{i} - 2)^{2} < 4.6 \bigg\}
     \end{align*}
\end{document}
