\documentclass[../probability-notes.tex]{subfiles}
\begin{document}
\section{Convolutions} \label{convolution}
Convolution operations are defined for both CDF and PDF/PMFs. Let $X$ and $Y$ be random independent variables, then
\begin{alignat*}{2}
    F_{X+Y}(x) &= F_{X} * F_{Y} &= \int_{\mathbb{R}} F_{X}(x-y) dF_{Y}(y)\\
    p_{X+Y}(x) &= p_{X} * p_{Y} &= \int_{\mathbb{R}} p_{X}(x-y) p_{Y}(y) dy
\end{alignat*}

We can extend the idea to $n$ independent variables as
\begin{align*}
    F_{X}^{n*} = F_{X} * \cdots * F_{X} \text{ $n$ times}
\end{align*}
It has the following properties for positive random variable $X_{i}$s
\begin{enumerate}
    \item
    \begin{align*}
        F_{X}^{n*}(x) \leq F_{X}^{n}(x)
    \end{align*}
    This can be proven as
    \begin{align*}
        P(X_{1} + \cdots + X_{n} \leq x) &\leq P(X_{1} \leq x, \ldots, X_{n} \leq x)\\
        P(X_{1} + \cdots + X_{n} \leq x) &\leq \prod_{i=1}^{n} P(X \leq x) \text{ by independence}\\
        \text{or, } F_{X}^{n*}(x) &\leq F_{X}^{n}(x)
    \end{align*}

    \item
    \begin{align*}
        F_{X}^{n*}(x) \geq F_{X}^{n+1}(x)
    \end{align*}
    which follows immediately from the fact that
    \begin{align*}
        P(X_{1} + \cdots + X_{n} \leq x) &\geq P(X_{1} \leq x, \ldots, X_{n+1} \leq x)\\
    \end{align*}
    since the volume of the regions denoting the sums will be lower in the higer dimensions. This can be quickly verified by considering $X_{1} \leq 1$ and $X_{1} + X_{2} <= 1$.
\end{enumerate}
\end{document}
