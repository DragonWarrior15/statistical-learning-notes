\documentclass[../probability-notes.tex]{subfiles}
\begin{document}
    %%%%%%%%%%%%%%%%%%%%%%%%%%%%%%%%%%%%%%%%%%%%%%%%%%%%%%%%%%%%%%%%%%%%%%%%%%%
    \chapter{Linear Regression}
    We are given pairs of data $(x_{1},y_{1}), (x_{2},y_{2}), \ldots, (x_{n},y_{n})$ (all independent) where we assume that $x$ and $y$ are governed by the linear relation
    \begin{align*}
        y \approx \theta_{0} + \theta_{1}x
    \end{align*}
    The aim is to determine the model which is parametric consisting of two parameters $\theta_{0}$ and $\theta_{1}$. We find it using the least squares estimate, i.e., minimizing
    \begin{align*}
        \minimize_{\theta_{0}, \theta_{1}} \sum_{i=1}^{n} (y_{i} - \theta_{0} - \theta_{1}x)^{2}
    \end{align*}
    The true model also includes noise and is given by
    \begin{align*}
        Y_{i} = \theta_{0} + \theta_{1}X_{i} + W_{i}
    \end{align*}
    where we assume the noise $W_{i} \sim \mathcal{N}(0, \sigma^{2})$ and is independently and identically distributed. Observing some $X$ and $Y$ is same as observing the noise.
    \begin{align*}
        P(X=x,Y=y) &= P(W=y-\theta_{0}-\theta_{1}x) = \frac{1}{\sqrt{2\pi \sigma_{2}}} \exp\bigg(-\frac{(y-\theta_{0}-\theta_{1}x)^{2}}{2 \sigma^{2}}\bigg)\\
        P(X_{1}=x_{1},Y_{1}=y_{1}, \ldots, X_{n}=x_{n},Y_{n}=y_{n}) &= \prod_{i=1}^{n} P(X_{1}=x_{i},Y_{i}=y_{i})\\
        &= \prod_{i=1}^{n} W_{i} = \prod_{i=1}^{n} \frac{1}{\sqrt{2\pi\sigma_{2}}} \exp\bigg(-\frac{(y_{i}-\theta_{0}-\theta_{1}x_{i})^{2}}{2\sigma^{2}}\bigg)
    \end{align*}
    Maximizing the above product is maximizing the likelihood of the occurrence of the data under the model parameters $\theta_{0}$ and $\theta_{1}$. Since taking log will not change the maxima, we usually maximize the log likelihood
    \begin{align*}
        \maximize_{\theta_{0}, \theta_{1}} \prod_{i=1}^{n} \frac{1}{\sqrt{2\pi\sigma^{2}}} \exp(-\frac{(y_{i}-\theta_{0}-\theta_{1}x_{i})^{2}}{2\sigma^{2}}) = \minimize_{\theta_{0}, \theta_{1}} \sum_{i=1}^{n} (y_{i}-\theta_{0}-\theta_{1}x_{i})^{2}
    \end{align*}

    We can take derivatives with respect to the parameters of the above function to get the estimate for the parameters as
    \begin{align*}
        \bar{x} &= \frac{1}{n} \sum_{i=1}^{n} x_{i} \text{,}\quad \bar{y} = \frac{1}{n} \sum_{i=1}^{n} y_{i}\\
        \hat{\theta}_{1} &= \frac{\sum_{i=1}^{n} (x_{i} - \bar{x}) (y_{i} - \bar{y})}{\sum_{i=1}^{n}(x_{i} - \bar{x})^{2}} = \frac{E[(X-\overline{X})(Y-\overline{Y})]}{E[(X-\overline{X})^{2}]} = \frac{Cov(X,Y)}{Var(X)}\\
        \hat{\theta}_{0} &= \bar{y} - \hat{\theta_{1}} \bar{x}
    \end{align*}

    The above formulae can also be derived if the additives are a function of $X$. Since the linear relationship will still be respected and the loglikelihood can be maximized to get the estimates of the parameters.\newline

    Some useful notation
    \begin{alignat*}{2}
        S_{xY} &= \sum_{i=1}^{n} (x_{i} - \bar{x})(Y_{i} - \overline{Y}) &= (\sum_{i=1}^{n}x_{i}Y_{i}) - n\bar{x}\overline{Y}\\
        S_{xx} &= \sum_{i=1}^{n} (x_{i} - \bar{x})^{2} &= (\sum_{i=1}^{n} x_{i}^{2}) - n\bar{x}^{2}\\
        S_{YY} &= \sum_{i=1}^{n} (Y_{i} - \overline{Y})^{2} &= (\sum_{i=1}^{n} Y_{i}^{2}) - n\overline{Y}^{2}
    \end{alignat*}

    %%%%%%%%%%%%%%%%%%%%%%%%%%%%%%%%%%%%%%%%%%%%%%%%%%%%%%%%%%%%%%%%%%%%%%%%%%%
    \section{Mean and Variance of Coefficients}
    First note that
    \begin{gather*}
        E[Y_{i}] = E[\theta_{0} + \theta_{1}X_{i} + W_{i}] = \theta_{0} + \theta_{1}X_{i}\\
        E[\overline{Y}] = (\sum_{i=1}^{n} E[Y_{i}])/n = \theta_{0} + \theta_{1}\overline{X}\\
        Var(Y_{i}) = \sigma_{2}
    \end{gather*}
    Thus,
    \begin{align*}
        E[\hat{\theta}_{1}] &= E\bigg[ \frac{\sum_{i=1}^{n} (x_{i} - \overline{x}) (Y_{i} - \overline{Y})}{\sum_{i=1}^{n}(x_{i} - \overline{x})^{2}} \bigg]\\ 
        &= E\bigg[ \frac{\sum_{i=1}^{n} (x_{i} - \overline{x}) (E[Y_{i}] - E[\overline{Y}])}{\sum_{i=1}^{n}(x_{i} - \overline{x})^{2}} \bigg]\\
        &= \theta_{1}\\
        E[\hat{\theta}_{0}] &= E[\overline{Y} - \hat{\theta_{1}} \bar{x}] = \theta_{0}
    \end{align*}
    meaning that our estimates of the parameters are unbiased and their error will equal the variance
    \begin{align*}
        Var(\hat{\theta}_{1}) &= Var \bigg( \frac{\sum_{i=1}^{n} (x_{i} - \overline{x}) (Y_{i} - \overline{Y})}{\sum_{i=1}^{n}(x_{i} - \overline{x})^{2}} \bigg)
        = Var \bigg( \frac{\sum_{i=1}^{n} (x_{i} - \overline{x})Y_{i}}{\sum_{i=1}^{n}(x_{i} - \overline{x})^{2}} \bigg)\\
        &= \frac{1}{(\sum_{i=1}^{n}(x_{i} - \overline{x})^{2})^{2}} \sum_{i=1}^{n} (x_{i} - \overline{x})^{2} Var(Y_{i})
        = \frac{\sigma^{2}}{\sum_{i=1}^{n}(x_{i} - \overline{x})^{2}}\\
        Var(\hat{\theta}_{0}) &= Var(\overline{Y} - \hat{\theta_{1}} \bar{x})
        = Var \bigg( \sum_{i=1}^{n} \bigg( \frac{1}{n} - \frac{\bar{x}(x_{i} - \bar{x})}{\sum_{i=1}^{n}(x_{i} - \bar{x})^{2}} \bigg) \bigg)\\
        &= \frac{\sigma^{2}}{n^{2}} \bigg( \sum_{i=1}^{n} \bigg( \frac{\sum_{i=1}^{n}x_{i}^{2} - n\bar{x}x_{i}}{\sum_{i=1}^{n}x_{i}^{2} - n\bar{x}^{2}} \bigg)^{2} \bigg)
        = \frac{\sigma^{2}}{n^{2} (\sum_{i=1}^{n}x_{i}^{2} - n\bar{x}^{2})^{2}} (n(\sum_{i=1}^{n})^{2} - n^{2}\bar{x}^{2}(\sum_{i=1}^{n})^{2})\\
        &= \sigma^{2} \frac{\sum_{i=1}^{n} x_{i}^{2}}{n\big((\sum_{i=1}^{n} x_{i}^{2}) - n\bar{x}^{2} \big)}
    \end{align*}
    because both the estimators are linear combinations of independent identically distributed normal random variables $Y_{i}s$, and the variance of linear combination of independent random variables is simply the sum of variances multiplied by squares of coefficients.\newline

    Thus, \textbf{$\hat{\theta_{0}}$ and $\hat{\theta}_{1}$ are both normally distributed random variables.} with the following distributions

    \begin{align*}
        \hat{\theta}_{1} &\sim \mathcal{N}\bigg(\theta_{1}, \frac{\sigma^{2}}{\sum_{i=1}^{n}(x_{i} - \overline{x})^{2}} \bigg)\\
        \hat{\theta}_{0} &\sim \mathcal{N}\bigg(\theta_{0}, \sigma^{2} \frac{\sum_{i=1}^{n} x_{i}^{2}}{n\big((\sum_{i=1}^{n} x_{i}^{2}) - n\bar{x}^{2} \big)} \bigg)
    \end{align*}


    %%%%%%%%%%%%%%%%%%%%%%%%%%%%%%%%%%%%%%%%%%%%%%%%%%%%%%%%%%%%%%%%%%%%%%%%%%%
    \section{Distribution of Residual}
    Residuals and the $SS_{R}$ are defined as
    \begin{align*}
        R &= Y - (\hat{\theta}_{0} + \hat{\theta}_{1}X)\\
        SS_{R} &= \sum_{i=1}^{n} R_{i}^{2} = \sum_{i=1}^{n} (Y - \hat{\theta}_{0} - \hat{\theta}_{1}X)^{2}\\
        &= \frac{S_{xx}S_{YY} - S_{xY}^{2}}{S_{xx}}
    \end{align*}

    $SS_{R}$ is itself a random variable and it can be shown that
    \begin{align*}
        \frac{SS_{R}}{\sigma^{2}} \sim \chi_{n-2}^{2}\\
        E[\frac{SS_{R}}{\sigma^{2}}] = n - 2\\
        E[\frac{SS_{R}}{n-2}] = \sigma^{2}\\
    \end{align*}
    since $SS_{R}/\sigma^{2}$ is the sum of squares of normally distributed variables ($E[Y] = \theta_{0} + \theta_{1}X$) and two degrees of freedoms are already taken up by the coefficients. Further, $SS_{R}$ is an unbiased estimator of the variance of the error terms $\sigma^{2}$, and is also independent of the coefficients.

    %%%%%%%%%%%%%%%%%%%%%%%%%%%%%%%%%%%%%%%%%%%%%%%%%%%%%%%%%%%%%%%%%%%%%%%%%%%
    \section{Inferences Concerning Coefficients}
    We are most interseted in checking whether a coefficient has an effect or not
    \begin{align*}
        H_{0}: \theta_{1} = 0 \quad \text{versus} \quad H_{1}: \theta_{1} \neq 0
    \end{align*}

    We know from above derivations that
    \begin{align*}
        \frac{\hat{\theta}_{1} - \theta_{1}}{\sigma^{2} / S_{xx}} \sim \mathcal{N}(0, 1)\\
        \frac{SS_{R}}{\sigma^{2}} \sim \chi_{n-2}^{2}
    \end{align*}
    and both the random variables are independent of each other. Hence their division is t-distributed random variable and when $H_{0}$ is true, $\theta_{1} = 0$
    \begin{align*}
        \frac{\sqrt{S_{xx}}\hat{\theta}_{1}/\sigma}{\sqrt{\frac{SS_{R}}{\sigma^{2} (n-2)}}} = \hat{\theta}_{1}\sqrt{\frac{(n-2)S_{xx}}{SS_{R}}} = TS \sim t_{n-2}
    \end{align*}
    We do this since we do not know the exact value of $\sigma^{2}$ and need to eliminate it with a sample derived version. The hypothesis test at significance level $\alpha$ simply becomes
    \begin{alignat*}{4}
        \text{Reject\quad} &H_{0} \text{\quad if \quad} &\lvert TS \rvert &> &t_{\alpha/2, n-2}\\
        \text{Accept\quad} &H_{0} \text{\quad if\quad} &\vert TS \rvert &\leq &t_{\alpha/2, n-2}
    \end{alignat*}
    which can be converted to a \emph{p-value} using the $TS$ and t-distribution. A small \emph{p-value} will lead to rejection of $H_{0}$ meaning that the data provides evidence of a relationship between dependent and independent variables.\newline

    A confidence interval for $\theta_{1}$ at $1-\alpha$ confidence can be obtained as follows
    \begin{gather*}
        P(-t_{\alpha/2, n-2} < (\hat{\theta}_{1} - \theta_{1})\sqrt{\frac{(n-2)S_{xx}}{SS_{R}}} < t_{\alpha/2, n-2}) = 1-\alpha\\
        \text{Confidence Interval is} \quad \bigg(\hat{\theta}_{1} - t_{\alpha/2, n-2}\sqrt{\frac{SS_{R}}{(n-2)S_{xx}}} < \theta_{1} < \hat{\theta}_{1} + t_{\alpha/2, n-2}\sqrt{\frac{SS_{R}}{(n-2)S_{xx}}} \bigg)
    \end{gather*}

    The hypothesis test for $\theta_{0}$ can be done in the exact same manner as $\theta_{1}$ by considering the following test statistic
    \begin{align*}
        TS = (\hat{\theta}_{1} - \theta_{1})\sqrt{\frac{n(n-2)S_{xx}}{(\sum_{i=1}^{n} x_{i}^{2})SS_{R}}} \sim t_{n-2}
    \end{align*}


    %%%%%%%%%%%%%%%%%%%%%%%%%%%%%%%%%%%%%%%%%%%%%%%%%%%%%%%%%%%%%%%%%%%%%%%%%%%
    \section{Inferences Concerning Mean Response}\label{sec:infer_mean_resp}
    For any new point $x_{0}$, the unbiased estimator for the response is
    \begin{align*}
        y_{0} &= \hat{\theta}_{0} + \hat{\theta}_{1}x_{0}\\
        E[y_{0}] &= E[\hat{\theta}_{0}] + E[\hat{\theta}_{1}]E[x_{0}] = \theta_{0} + \theta_{1}x_{0}
    \end{align*}
    To get the distribution of this mean response, note that
    \begin{align*}
        Y_{0} &= \hat{\theta}_{0} + \hat{\theta}_{1}x_{0} = \overline{Y} - \hat{\theta}_{1}\bar{x} + \hat{\theta}_{1}x_{0}\\
        &= \frac{1}{n}\sum_{i=1}^{n} Y_{i} + (x_{0} - \bar{x})\frac{\sum_{i=1}^{n} (x_{i} - \bar{x})Y_{i}}{\sum_{i=1}^{n} (x-\bar{x})^{2}}\\
        &= \sum_{i=1}^{n} \bigg( \frac{1}{n} + \frac{(x_{i} - \bar{x})(x_{0} - \bar{x})}{S_{xx}} \bigg)Y_{i}
    \end{align*}
    which is a linear combination of independent normally distributed random variables $Y_{i}s$. Thus, the mean response is also a normally distributed random variable and we can get the confidence intervals by considering the mean and variance of this random variable
    \begin{align*}
        Var(\hat{\theta}_{0} + \hat{\theta}_{1}x_{0}) &= \sum_{i=1}^{n} \bigg( \frac{1}{n} + \frac{(x_{i} - \bar{x})(x_{0} - \bar{x})}{S_{xx}} \bigg)^{2}Var(Y_{i})\\
        \hat{\theta}_{0} + \hat{\theta}_{1}x_{0} &\sim \mathcal{N}\bigg(\theta_{0} + \theta_{1}x_{0}, \sigma^{2} \bigg[ \frac{1}{n} + \frac{(x_{0} - \bar{x})^{2}}{S_{xx}} \bigg]\bigg)
    \end{align*}
    To eliminate $\sigma^{2}$,
    \begin{gather*}
        SS_{R}/\sigma^{2} \sim \chi_{n-2}^{2}\\
        \frac{(\hat{\theta}_{0} + \hat{\theta}_{1}x_{0}) - (\theta_{0} + \theta_{1}x_{0})}{\sigma^{2} \bigg[ \frac{1}{n} + \frac{(x_{0} - \bar{x})^{2}}{S_{xx}} \bigg]} \div \sqrt{\frac{SS_{R}}{(n-2)\sigma^{2}}} \sim t_{n-2}
    \end{gather*}
    and the confidence intervals for confidence $1-\alpha$ become
    \begin{align*}
        (\hat{\theta}_{0} + \hat{\theta}_{1}x_{0}) \pm t_{\alpha/2, n-2} \sqrt{\bigg( \frac{1}{n} + \frac{(x_{0} - \bar{x})^{2}}{S_{xx}} \bigg) \bigg( \frac{SS_{R}}{n-2}\bigg)}
    \end{align*}


    %%%%%%%%%%%%%%%%%%%%%%%%%%%%%%%%%%%%%%%%%%%%%%%%%%%%%%%%%%%%%%%%%%%%%%%%%%%
    \section{Inferences Concerning Future Response}
    The above section \ref{sec:infer_mean_resp} discussed the distribution of the mean response. In many scenarios, we are interested in the distribution of the actual response $Y$ at input $x_{0}$, which takes the noise into account as well. We note
    \begin{align*}
        Y_{0} &\sim \mathcal{N}(\theta_{0} + \theta_{1}x_{0}, \sigma^{2})\\
        \hat{\theta}_{0} + \hat{\theta}_{1}x_{0} &\sim \mathcal{N}\bigg(\theta_{0} + \theta_{1}x_{0}, \sigma^{2} \bigg[ \frac{1}{n} + \frac{(x_{0} - \bar{x})^{2}}{S_{xx}} \bigg]\bigg)\\
        Y_{0} - \hat{\theta}_{0} - \hat{\theta}_{1}x_{0} &\sim \mathcal{N}\bigg(0, \sigma^{2}\bigg( 1 + \frac{1}{n} + \frac{(x_{0} - \bar{x})^{2}}{S_{xx}} \bigg)\bigg)
    \end{align*}

    Now we utilise the distribution of $SS_{R}$ to eliminate $\sigma^{2}$ and get to the t-distribution
    \begin{align*}
        \frac{Y_{0} - \hat{\theta}_{0} - \hat{\theta}_{1}x_{0}}{\sigma\sqrt{1 + \frac{1}{n} + \frac{(x_{0} - \bar{x})^{2}}{S_{xx}}}} \div \sqrt{\frac{SS_{R}}{(n-2)\sigma^{2}}} \sim t_{n-2}
    \end{align*}
    and the \textbf{prediction} interval for the response (not mean response is) at $1-\alpha$ confidence
    \begin{align*}
        (\hat{\theta}_{0} + \hat{\theta}_{1}x_{0}) \pm t_{\alpha/2, n-2} \sqrt{\bigg( 1+ \frac{1}{n} + \frac{(x_{0} - \bar{x})^{2}}{S_{xx}} \bigg) \bigg( \frac{SS_{R}}{n-2}\bigg)}
    \end{align*}

    Note that \textbf{prediction interval is the interval where we expect the value of a random variable to lie, whereas the confidence interval is the one where the value of a parameter estimate to lie.}

    
    %%%%%%%%%%%%%%%%%%%%%%%%%%%%%%%%%%%%%%%%%%%%%%%%%%%%%%%%%%%%%%%%%%%%%%%%%%%
    \section{Coefficient of Determination}
    Let's consider the variation in respone Y
    \begin{align*}
        S_{Y} = \sum_{i=1}^{n} (Y_{i} - \overline{Y})^{2}
    \end{align*}
    and the variation in the response after removing the effect of inputs
    \begin{align*}
        SS_{R} = \sum_{i=1}^{n} (Y_{i} - \theta_{0} - \theta_{1}x_{0})^{2}
    \end{align*}
    and thus,
    \begin{align*}
        S_{YY} - SS_{R}
    \end{align*}
    is the variation explained by the inputs. We define $R^{2}$ as
    \begin{align*}
        R^{2} = \frac{S_{YY} - SS_{R}}{S_{YY}} = 1 - \frac{SS_{R}}{S_{YY}}
    \end{align*}
    \textbf{$R^{2}$ is the proportion of total variance explained by the inputs. A value close to 1 implies most of the variance is explained by the inputs whereas 0 means little variance is explained by inputs.}\newline

    It can also be shown that the absolute value of correlation coefficient between $x$ and $Y$ equals the coefficient of determination. Thus, we know the value of $R^{2}$ for simple linear regression directly by $r$.


    %%%%%%%%%%%%%%%%%%%%%%%%%%%%%%%%%%%%%%%%%%%%%%%%%%%%%%%%%%%%%%%%%%%%%%%%%%%
    \section{Weighted Least Squares}
    Suppose we know that the variance of $Y$ is dependent on $Y$ itself in the form $Var(Y_{i}) \propto \sigma^{2}/w_{i}$, i.e., the weights are known only upto a constant. In this case, we minimize the weighted least squares to obtain the coefficients
    \begin{align*}
        \minimize_{\theta_{0}, \theta_{1}} \sum_{i=1}^{n} w_{i}(Y_{i} - \theta_{0} - \theta_{1}x_{i})^{2}
    \end{align*}

\end{document}
