\documentclass[../probability-notes.tex]{subfiles}
\begin{document}
    %%%%%%%%%%%%%%%%%%%%%%%%%%%%%%%%%%%%%%%%%%%%%%%%%%%%%%%%%%%%%%%%%%%%%%%%%%%
    \chapter{Moment Generating Function}
    Moment generating function is defined as the following for all values of $t$
    \begin{align*}
        \phi (t) = E[e^{tX}] = \begin{cases} \sum_{x} e^{tx} p_{X}(x) &\mbox{for discrete case}\\
        \int_{-\inf}^{\inf} e^{tx} f_{X}(x) &\mbox{for continuous case} \end{cases} 
    \end{align*}
    This function is called the moment generating function because all the moments of the random variable $X$ can be obtained by successively differentiating the function $\phi(t)$.\newline

    \begin{align*}
        \phi^{\prime}(t) &= \frac{d}{dt} E[e^{tX}]\\
        &= E[\frac{d}{dt} e^{tX}]\\
        &= E[Xe^{tX}]\\
        \text{mean} &= E[X]\\ 
        &= \phi^{\prime}(0)
    \end{align*}

    Continuing in a similar fashion,
    \begin{align*}
        \phi^{\prime\prime}(t) &= \frac{d}{dt} E[Xe^{tX}]\\
        &= E[\frac{d}{dt}Xe^{tX}]\\
        &= E[X^{2}e^{tX}]\\
        \text{variance} &= \phi^{\prime\prime}(0)\\
        &= E[X^{2}]
    \end{align*}
    In general, for any $n > 0$, the $n^{th}$ derivative will give the $n^{th}$ moment
    \begin{align*}
        \phi^{n}(0) = E[X^{n}]
    \end{align*}

    There exists a one to one correspondence between the moment generating function and the distribution function of a random variable, similar to Lagrangian multipliers.

    %%%%%%%%%%%%%%%%%%%%%%%%%%%%%%%%%%%%%%%%%%%%%%%%%%%%%%%%%%%%%%%%%%%%%%%%%%%
    \section{Moment Generating Function for Sum of Independent RV}
    An important property is in the context of sum of two or more random variables. The \textbf{moment generating of sum of two independent random variables is simply the product of the moment generating functions of the two individual random variables}
    \begin{align*}
        \phi_{X+Y}(t) &= E[e^{t(X+Y)}]\\
        &= E[e^{tX} e^{tY}]\\
        &= E[e^{tX}] E[e^{tY}]\\
        \Aboxed{\phi_{X+Y}(t) &= \phi_{X}(t) \phi_{Y}(t)} \text{\; for independent random variables}
    \end{align*}
\end{document}