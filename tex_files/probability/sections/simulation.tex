\documentclass[../probability-notes.tex]{subfiles}
\begin{document}
    %%%%%%%%%%%%%%%%%%%%%%%%%%%%%%%%%%%%%%%%%%%%%%%%%%%%%%%%%%%%%%%%%%%%%%%%%%%
    \section{Simulation, Random Numbers, Permutation Tests}
    %%%%%%%%%%%%%%%%%%%%%%%%%%%%%%%%%%%%%%%%%%%%%%%%%%%%%%%%%%%%%%%%%%%%%%%%%%%
    \subsection{Random Numbers}
    We can generate random numbers using the following equation
    \begin{align*}
        x_{n+1} = (ax_{n} + c) mod(m)
    \end{align*}
    $x_{n}$ takes the values $1,2,\ldots, m-1$ and we take $x_{n}/m$ as the pseudo random number, which is uniformly distributed between $(0,1)$ for suitable choice of $a, c, m$.\newline

    
    %%%%%%%%%%%%%%%%%%%%%%%%%%%%%%%%%%%%%%%%%%%%%%%%%%%%%%%%%%%%%%%%%%%%%%%%%%%
    \subsubsection{Permutation of Integers}
    Suppose we want to generate a permutation of integers from $1, 2, \ldots, n$ such that each of the permutations is equally likely. Assuming we have a uniform random generator $U$ with us, 
    \begin{align*}
        P(Int(kU) + 1 = i) &= P(Int(kU) = i-1) = P(i-1 \leq kU < i)\\
        &= P(\frac{i-1}{k} \leq U < \frac{i}{k}) = \frac{1}{k}
    \end{align*}
    which gives us randomly generated random integers between $1$ and $k$ with equal probability. An easy way to generate permutation is
    \begin{enumerate}
        \item Choose a permutation $r_{1}, r_{2}, \ldots, r_{n}$ which can just be $r_{j} = j$
        \item Let $k = n$
        \item Choose a random number $U$ and let $I = Int(kU) + 1$
        \item Interchange numbers at position $k$ and $I$
        \item $k = k-1$
        \item if $k > 1$ goto step 3 else return permutation
    \end{enumerate}

    The above algorithm can also be used to get a random subset of size $r$ from a set $1, \dots, k$ by simply running the algorithm till $k = r$ since the elements in the last $r$ positions can be selected. For $r > n/2$, we find the $k=n-r$ elements not in the subset.


    %%%%%%%%%%%%%%%%%%%%%%%%%%%%%%%%%%%%%%%%%%%%%%%%%%%%%%%%%%%%%%%%%%%%%%%%%%%
    \subsection{Bootstrap Method}


    %%%%%%%%%%%%%%%%%%%%%%%%%%%%%%%%%%%%%%%%%%%%%%%%%%%%%%%%%%%%%%%%%%%%%%%%%%%
    \subsection{Generating Discrete Random Variables}
    Suppose we want to generate the random variable $X$ with probability mass function
    \begin{align*}
        P(X = x_{i}) = p_{i}, i = 1, 2, \ldots, n\; \sum_{i=1}^{n} p_{i} = 1 
    \end{align*}
    Then using a uniform random generator $U$, we can generate the discrete random variable using
    \begin{align*}
        X = x_{i} \quad \text{if} \quad p_{1} + p_{2} +\cdots + p_{i-1} \leq U < p_{1} + p_{2} +\cdots + p_{i} 
    \end{align*}
    i.e., we divide the number line at points $p_{1}, p_{1}+p_{2}, \ldots, 1$ and choose the $i^{th}$ interval such that $U$ falls in that interval. This algorithm is valid since
    \begin{gather*}
        P(a \leq U < b) = b-a\\
        P(\sum_{j=1}^{i-1}p_{j} \leq U < \sum_{j=1}^{i}p_{j}) = p_{i}
    \end{gather*}

    This method is known as \emph{discrete inverse transform method}.\newline


    %%%%%%%%%%%%%%%%%%%%%%%%%%%%%%%%%%%%%%%%%%%%%%%%%%%%%%%%%%%%%%%%%%%%%%%%%%%
    \subsubsection{Binomial Random Variable}
    To generate a Bernoulli random variable, we simply select $X = 1$ if $U < p$ otherwise $X = 0$. Similarly a binomial random variable can be generated using individual Bernoulli variables as described. A more efficient method is to use the inverse transform method. For number of successes $0, 1, 2, \ldots, n$, we must calculate the probability mass function. This can be done efficiently using recursion
    \begin{gather*}
        p_{i} = P(X = i) = \binom{n, i} p^{i} (1-p)^{n-i}\\
        \frac{p_{i+1}}{p_{i}} = \frac{n-i}{i+1} \frac{p}{1-p}
    \end{gather*}

    The algorithm is then simply
    \begin{enumerate}
        \item Assign $i = 0, P = p_{0} = (1-p)^{n}, F = P, b = p/(1-p)$
        \item Generate random number $U \in (0, 1)$
        \item if $U \leq F, X = i$, stop else continue
        \item Update $P$ to get $p_{i+1}$, $P = Pb\frac{n-i}{i+1}$
        \item Update the cumulative probability $F = F + P$
        \item increase $i = i + 1$, goto 3
    \end{enumerate}

    The average number of iterations taken by the algorithm $= E[X + 1] = np + 1$ since total values checked are $n + 1$.

\end{document}