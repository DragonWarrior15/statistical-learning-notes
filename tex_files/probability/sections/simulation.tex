\documentclass[../probability-notes.tex]{subfiles}
\begin{document}
    %%%%%%%%%%%%%%%%%%%%%%%%%%%%%%%%%%%%%%%%%%%%%%%%%%%%%%%%%%%%%%%%%%%%%%%%%%%
    \section{Generating Random Numbers}
    %%%%%%%%%%%%%%%%%%%%%%%%%%%%%%%%%%%%%%%%%%%%%%%%%%%%%%%%%%%%%%%%%%%%%%%%%%%
    \subsection{Random Numbers}
    We can generate random numbers using the following equation
    \begin{align*}
        x_{n+1} = (ax_{n} + c) mod(m)
    \end{align*}
    $x_{n}$ takes the values $1,2,\ldots, m-1$ and we take $x_{n}/m$ as the pseudo random number, which is uniformly distributed between $(0,1)$ for suitable choice of $a, c, m$.\newline

    
    %%%%%%%%%%%%%%%%%%%%%%%%%%%%%%%%%%%%%%%%%%%%%%%%%%%%%%%%%%%%%%%%%%%%%%%%%%%
    \subsubsection{Permutation of Integers}
    Suppose we want to generate a permutation of integers from $1, 2, \ldots, n$ such that each of the permutations is equally likely. Assuming we have a uniform random generator $U$ with us, 
    \begin{align*}
        P(Int(kU) + 1 = i) &= P(Int(kU) = i-1) = P(i-1 \leq kU < i)\\
        &= P(\frac{i-1}{k} \leq U < \frac{i}{k}) = \frac{1}{k}
    \end{align*}
    which gives us randomly generated random integers between $1$ and $k$ with equal probability. An easy way to generate permutation is
    \begin{enumerate}
        \item Choose a permutation $r_{1}, r_{2}, \ldots, r_{n}$ which can just be $r_{j} = j$
        \item Let $k = n$
        \item Choose a random number $U$ and let $I = Int(kU) + 1$
        \item Interchange numbers at position $k$ and $I$
        \item $k = k-1$
        \item if $k > 1$ goto step 3 else return permutation
    \end{enumerate}

    The above algorithm can also be used to get a random subset of size $r$ from a set $1, \dots, k$ by simply running the algorithm till $k = r$ since the elements in the last $r$ positions can be selected. For $r > n/2$, we find the $k=n-r$ elements not in the subset.


    %%%%%%%%%%%%%%%%%%%%%%%%%%%%%%%%%%%%%%%%%%%%%%%%%%%%%%%%%%%%%%%%%%%%%%%%%%%
    \subsection{Generating Discrete Random Variables}
    Suppose we want to generate the random variable $X$ with probability mass function
    \begin{align*}
        P(X = x_{i}) = p_{i}, i = 1, 2, \ldots, n\; \sum_{i=1}^{n} p_{i} = 1 
    \end{align*}
    Then using a uniform random generator $U$, we can generate the discrete random variable using
    \begin{align*}
        X = x_{i} \quad \text{if} \quad p_{1} + p_{2} +\cdots + p_{i-1} \leq U < p_{1} + p_{2} +\cdots + p_{i} 
    \end{align*}
    i.e., we divide the number line at points $p_{1}, p_{1}+p_{2}, \ldots, 1$ and choose the $i^{th}$ interval such that $U$ falls in that interval. This algorithm is valid since
    \begin{gather*}
        P(a \leq U < b) = b-a\\
        P(\sum_{j=1}^{i-1}p_{j} \leq U < \sum_{j=1}^{i}p_{j}) = p_{i}
    \end{gather*}

    This method is known as \emph{discrete inverse transform method}.\newline


    %%%%%%%%%%%%%%%%%%%%%%%%%%%%%%%%%%%%%%%%%%%%%%%%%%%%%%%%%%%%%%%%%%%%%%%%%%%
    \subsubsection{Binomial Random Variable}
    To generate a Bernoulli random variable, we simply select $X = 1$ if $U < p$ otherwise $X = 0$. Similarly a binomial random variable can be generated using individual Bernoulli variables as described. A more efficient method is to use the inverse transform method. For number of successes $0, 1, 2, \ldots, n$, we must calculate the probability mass function. This can be done efficiently using recursion
    \begin{gather*}
        p_{i} = P(X = i) = \binom{n, i} p^{i} (1-p)^{n-i}\\
        \frac{p_{i+1}}{p_{i}} = \frac{n-i}{i+1} \frac{p}{1-p}
    \end{gather*}

    The algorithm is then simply
    \begin{enumerate}
        \item Assign $i = 0, P = p_{0} = (1-p)^{n}, F = P, b = p/(1-p)$
        \item Generate random number $U \in (0, 1)$
        \item if $U \leq F, X = i$, stop else continue
        \item Update $P$ to get $p_{i+1}$, $P = Pb\frac{n-i}{i+1}$
        \item Update the cumulative probability $F = F + P$
        \item increase $i = i + 1$, goto 3
    \end{enumerate}

    The average number of iterations taken by the algorithm $= E[X + 1] = np + 1$ since total values checked are $n + 1$.


    %%%%%%%%%%%%%%%%%%%%%%%%%%%%%%%%%%%%%%%%%%%%%%%%%%%%%%%%%%%%%%%%%%%%%%%%%%%
    \subsection{Generating Continuous Random Variables}
    To generate a random variable $X$, We utilise it's cumulative distribution function $F$. Note that $F$ is strictly increasing from $0$ to $1$ and the probability density $f$ and $F$ have a one to one mapping, i.e., $F$ has an inverse $F^{-1}$. So, to get $X$, we first generate $U \in (0, 1)$ and then $X = F^{-1}(U)$.\newline

    %%%%%%%%%%%%%%%%%%%%%%%%%%%%%%%%%%%%%%%%%%%%%%%%%%%%%%%%%%%%%%%%%%%%%%%%%%%
    \subsubsection{Exponential Random Variable}
    The distribution function $F = 1 - exp(-\lambda x)$. Then, the inverse is
    \begin{align*}
        x &= -\frac{1}{\lambda}log(1 - F)\\
        \text{or, }\quad X &= \frac{1}{\lambda}log(1 - U)
    \end{align*}
    and $X$ will follow the exponential distribution. Further, replacing $U$ with $1-U$ will still remain a uniform distribution, meaning $X$ will still be exponential.

    
    %%%%%%%%%%%%%%%%%%%%%%%%%%%%%%%%%%%%%%%%%%%%%%%%%%%%%%%%%%%%%%%%%%%%%%%%%%%
    \subsubsection{Normal Random Variable}
    Consider $X, Y \sim \mathcal{N}(0, 1)$. Then,
    \begin{align*}
        f_{XY}(x,y) = \frac{1}{2\pi}exp(-\frac{x^{2}+y^{2}}{2})
    \end{align*}
    is the joint distribution. $(X, Y)$ is a point on the cartesian plane and thus, will have an equivalent polar coordinate $(R, \Theta)$ implying $R^{2} = X^{2} + Y^{2}$ which is a $\chi_{2}^{2}$ variable. From section \ref{sec:rel_gamma_chi},
    \begin{align*}
       f_{R^{2}}(r) &= \frac{1}{2}exp(-\frac{r}{2})\\
       \text{and} \quad f_{XY}(x,y) &= \frac{1}{2\pi} exp(-\frac{r}{2}) \quad \text{when} \quad x^{2} + y^{2} = r
    \end{align*}

    Now, $R$ and $\Theta$ are independent variables and can be used to generate $X$ and $Y$. $\Theta$ is uniformly distributed in $[0, 2\pi]$. Taking the inverse of cumulative distribution of $R^{2}$ (Note that $f_{R^{2}}(r)$ is a distribution on $R^{2}$ and not $R$),
    \begin{align*}
        R^{2} &= -2log(1 - U_{1})\\
        \Theta &= 2\pi U_{2}
    \end{align*}
    where $U_{1}$ and $U_{2}$ are uniform random variables. Using the transformation back to cartesian coordinates from polar ones,
    \begin{align*}
        X &= \sqrt{-2log(1 - U_{1})}cos(2\pi U_{2})\\
        Y &= \sqrt{-2log(1 - U_{1})}sin(2\pi U_{2})
    \end{align*}
    This approach is called \emph{Box-Muller method}. To generate standard normals with mean $\mu$ and variance $\sigma^{2}$, we simply return $\mu + \sigma X, \mu+\sigma Y$.

    
    %%%%%%%%%%%%%%%%%%%%%%%%%%%%%%%%%%%%%%%%%%%%%%%%%%%%%%%%%%%%%%%%%%%%%%%%%%%
    \subsubsection{Normal Random Variable from Uniform Distribution}
    Using the central limit theorem, we can add up multiple uniform distributions to get a normal random variable.
    \begin{gather*}
        Y = \bigg(\sum_{i=1}^{12}X_{i} \bigg) - 6 \sim \mathcal{N}(0,1)
    \end{gather*}
    because the mean of 12 uniform random variables is $12 * 0.5 = 6$ and the variance of sum of 12 independent random variables is $12 \times (1-0)^{2}/12 = 1$. The same equation with 30 uniform distributions will become
    \begin{gather*}
        Y = \sqrt{\frac{2}{5}}\bigg[\bigg(\sum_{i=1}^{30}X_{i} \bigg) - 15\bigg] \sim \mathcal{N}(0,1)
    \end{gather*}
    and combining more variables will increase the accuracy of the approximation, but also increase computational time.
\end{document}