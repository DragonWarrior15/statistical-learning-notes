\documentclass[../../statistical_learning_notes.tex]{subfiles}
\begin{document}
%%%%%%%%%%%%%%%%%%%%%%%%%%%%%%%%%%%%%%%%%%%%%%%%%%%%%%%%%%%%%%%%%%%%%%%%%%%
\section{Maximum A Priori (MAP)}
Similar to MLE, we have another estimator method which evaluates $\theta$ using the posterior probability of $\theta$ given $X$
\begin{align*}
    \hat{\theta}_{MAP} &= \max_{\theta} P(\theta|X) = \max_{\theta} \frac{P(X|\theta)P(\theta)}{P(X)}\\
    &= \max_{\theta}P(X|\theta)P(\theta)
\end{align*}

since $P(X)$ is just a normalization factor and thus constant. However, this sufferes from multiple problems
\begin{enumerate}
    \item Not invariant to reparametrization. Hence solution could change with algebraic manipulation.
    \item Can't be used as a prior since we will be looking at delta functions in the prior then.
    \item Finds untypical points (bumps) that are not smooth points in the curve.
    \item Cannot compute confidence intervals since we only get point estimates.
\end{enumerate}
\end{document}