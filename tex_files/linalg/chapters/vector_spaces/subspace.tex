\documentclass[../../linear_algebra.tex]{subfiles}
\begin{document}
%%%%%%%%%%%%%%%%%%%%%%%%%%%%%%
\section{Subspace}
A subset $U$ of vector space $\setv$ is called a subspace if $U$ is also a vector space on the same definitions of addition and scalar multiplication as on $\setv$.\newline

\subsection{Conditions for subspace}
A subset $U$ of vector space $\setv$ is a subspace of $\setv$ if and only if $U$ satisfies the following three conditions
\begin{itemize}
    \item Additive Identity\newline
    The additive identity $0 \in U$
    \item Closed under Addition\newline
    If $u,v \in U$, then $u + v \in U$
    \item Closed under Multiplication\newline
    If $u \in U$, then $\lambda u \in u$ for all $\lambda \in \field$
\end{itemize}

With these conditions, empty sets are not a vector subspace of $\setv$ and must contain at least one element to qualify as a vector space. The smalles subspace of $\setv$ is $\{ 0 \}$ and the largest subspace is $\setv$ itself.\newline

It is easy to verify that the subspaces of $\real^{2}$ are $\{ 0\}$, $\real^{2}$ and all lines through the origin ($0$). For $\real^{3}$, the subspaces will be $\{ 0\}$, $\real^{3}$, the set of all lines through origin, and the set of all planes through the origin.

\subsection{Sum of Subspaces}
For subsets $U_{1}, U_{2}, \ldots U_{m}$ of $\setv$, the sum denoted by $U_{1} + U_{2} + \cdots + U_{m}$ is the set of all possible sums of elements of all the $m$ subsets. More precisely
\begin{align*}
    U_{1} + U_{2} + \cdots + U_{m} = \{u_{1} + u_{2} + \cdots + u_{m} | u_{1} \in U_{1}, u_{2} \in U_{2}, \ldots, u_{m} \in U_{m} \}
\end{align*}

Furthermore, this sum of subspaces is the smallest subspace containing all the subspaces $U_{1}, \ldots, U_{m}$.

\subsection{Direct Sum}
For subsets $U_{1}, U_{2}, \ldots, U_{m}$, the direct sum is denoted by
\begin{align*}
    U_{1} + U_{2} + \cdots + U_{m} = U_{1} \oplus U_{2} \oplus \cdots \oplus U_{m} 
\end{align*}

The sum is a direct sum when any element of the direct sum can be expressed as the sum of elements of the subsets in a unique way.
\begin{align*}
    U_{1} \oplus U_{2} \oplus \cdots \oplus U_{m} = u_{1} + u_{2} + \cdots + u_{m}
\end{align*}
where $u_{i} \in U_{i}$ and there is a unique way to write this sum.\newline

As an example, let
\begin{align*}
    U_{1} &= {(x,y,0) | x,y \in \field}\\
    U_{2} &= {(0,0,z) | z \in \field}\\
    U_{1}, U_{2} &\in \field^{3}\\
    \text{Then,} \field^{3} &= U_{1} \oplus U_{2}
\end{align*}

$U_{1} + \cdots + U_{m}$ is a direct sum if and only if there is a way to write $0$ as a sum $u_{1} + \cdots + u_{m}$ such that all the $u_{i}$ are $0$.\newline

The sum of two subspaces is a direct sum if and only if the intersection of those two subspaces is $0$, i.e., for $U, W \in \setv$, we have $U \cap W = \{ 0 \}$
\end{document}