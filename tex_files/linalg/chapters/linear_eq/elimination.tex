\documentclass[../../linear_algebra.tex]{subfiles}
\begin{document}
%%%%%%%%%%%%%%%%%%%%%%%%%%%%%%
\section{Gauss Jordan Elimination}
This method of solving linear equations uses three elementary matrix operations
\begin{itemize}
    \item Shuffle rows of the matrix
    \item Multiply a row with a non zero scalar
    \item Add or subtract scalar multiple of a row from another row
\end{itemize}

The purpose of elimination is to convert the matrix to a reduced row echelon form which has the following characteristics
\begin{itemize}
    \item All rows with only zero entries are at the bottom of the matrix
    \item The first non zero entry in a row called the leading entry or the pivot of each non zero row is to the right of the leading entry of the row above it
    \item The leading entry in any non zero row is 1
    \item All other entries in a column containing a leading 1 are zeros
\end{itemize}
The following are the steps to perform the elimination algorithm
\begin{enumerate}
    \item Swap the rows so that all rows with 0 entries are at the bottom of the matrix
    \item Swap the rows so that the row with the largest leftmost non zero entry is at the top
    \item Multiply the top row by a scalar so that the leftmost entry becomes 1
    \item Add or subtract multiples of the top row from all the other rows so that all the entries in the column containing the topmost row's leading entry apart from the topmost row become 0
    \item Repeat steps 2-4 for the next leftmost entry until all the leading entries are 1
    \item Swap the rows so that the leading entry of each nonzero row is to the right of the leading entry of the row above it
\end{enumerate}

As an example, consider solving the following system of equations
\begin{align*}
    x + 2y + z &= 2\\
    3x + 8y + z &= 12\\
    4y + z &= 2\\
    \text{or, } \begin{bmatrix}
    1 &2 &1\\
    3 &8 &1\\
    0 &4 &1\\
    \end{bmatrix}
    \begin{bmatrix}
        x\\y\\z
    \end{bmatrix} = \begin{bmatrix}
        2\\12\\2
    \end{bmatrix}
\end{align*}
The matrix we will work with during the algorithm is a composite matrix $A \lvert b$
\begin{align*}
    \begin{bmatrix}
    1 &2 &1 &\lvert &2\\
    3 &8 &1 &\lvert &12\\
    0 &4 &1 &\lvert &2\\
    \end{bmatrix}
\end{align*}
\end{document}
