\documentclass[../../time_series_notes.tex]{subfiles}
\begin{document}
%%%%%%%%%%%%%%%%%%%%%%%%%%%%%%
\section{Backward Shift Operators}
This is a very useful notation in the analysis of time series, especially ARIMA models. A single backward shift operator denotes the time series with a single lag
\begin{align*}
    BX_{t} = X_{t-1}
\end{align*}
where $X$ is the random variable denoting the time series. We can recursively apply this operator to get
\begin{align*}
    B^{2}X_{t} &= BX_{t-1} = X_{t-2}\\
    B^{k}X_{t} &= X_{t-k}
\end{align*}

This operator can be used in time series analysis to create polynomial like functions to convert from one time series to another.
\begin{align*}
    X_{t} &= Z_{t} + \theta_{1}Z_{t-1} + \theta_{2}Z_{t-2}\\
    &= Z_{t} + \theta_{1}BZ_{t} + \theta_{2}B^{2}Z_{t}\\
    &= (1 + \theta_{1}B + \theta_{2}B^{2})Z_{t}
\end{align*}


An inverse of this operator also exists such that
\begin{align*}
    B^{-1}B &= 1\\
    B^{-1}X_{t-1} &= X_{t}
\end{align*}
where $B^{-1}$ is called the \textbf{forward shift operator}.\newline

To make time series analysis easier, it is often the case that we will take difference of successive terms in order to induce the stationarity conditions. We conveniently define the \textbf{Differencing operator} as
\begin{align*}
    \nabla X_{t} &= (1-B)X_{t} = X_{t} - X_{t-1}\\
    \nabla^{2}X_{t} &= (1-B)^{2}X_{t} = (1 - 2B + B^{2})X_{t}\\
    \nabla^{2}X_{t} &= \nabla(\nabla X_{t}) = \nabla(X_{t} - X_{t-1})\\
    &= (X_{t} - X_{t-1}) - (X_{t-1} - X_{t-2}) = X_{t} - 2BX_{t} + B^{2}X_{t}\\
    \nabla^{d}X_{t} &= (1-B)^{d}X_{t} \numberthiseqn\label{eq:eq_arima_1}
\end{align*}
which illustrates how we can use the differencing operator to obtain successive differences. Higher powers of this operator can be expanded algebraically using the equation \eqref{eq:eq_arima_1}.\newline

Differencing is a technique frequently used to make the time series stationary. Usually a difference of order 1 or 2 suffices. In some cases, we may need to remove the seasonal trend in which case we will resort to order $m$ difference where $m$ is the length of a seasonal cycle.


\end{document}