\documentclass[11pt, a4paper, notitlepage]{report}

\usepackage{amsmath}
\usepackage{bm}
\usepackage[a4paper, margin=1in]{geometry}
\usepackage{amsfonts}
\usepackage{mathtools}
% \usepackage{tikz}
% \usetikzlibrary{automata, positioning}

\usepackage{pgfplots}
\pgfplotsset{width=5cm,compat=1.9}

% remove paragraph indent
\setlength{\parindent}{0em}

% hyperlinks for the table of contents
% \usepackage{color}
\usepackage[pdfencoding=auto, psdextra]{hyperref}
\hypersetup{
    colorlinks=true, % make the links colored
    linkcolor=blue, % color TOC links in blue
    urlcolor=red, % color URLs in red
    linktoc=all, % 'all' will create links for everything in the TOC
    bookmarksnumbered=true
}

% \setcounter{tocdepth}{4}
\usepackage[level=subsection]{bookmark}

% to insert images
\usepackage{graphicx}
\graphicspath{ {./images/} {../images/} {../../images/}}

\usepackage{subfiles}

% make the toc start on the same page as title
\makeatletter
\newcommand*{\tocontents}{\@starttoc{toc}}
\makeatother

\DeclareMathOperator*{\minimize}{minimize}
\DeclareMathOperator*{\maximize}{maximize}
\DeclareMathOperator*{\argmin}{argmin}
\DeclareMathOperator*{\argmax}{argmax}

% \newcommand{\summation}[3]{\sum_{#1 = #2}^{#3}}
% \renewcommand{\theequation}{\thesection.\arabic{equation}}
\newcommand{\numberthiseqn}{\addtocounter{equation}{1}\tag{\theequation}}

\begin{document}
    \pagenumbering{arabic}
    \title{Time Series Analysis Notes}
    \date{}
    \author{}
    \maketitle
    \pdfbookmark{\contentsname}{toc}
    \tocontents

    %%%%%%%%%%%%%%%%%%%%%%%%%%%%%%%%%%%%%%%%%%%%%%%%%%
    \subfile{chapters/characteristics/characteristics}
        \subfile{chapters/characteristics/formulations}
        \subfile{chapters/characteristics/measures}
        \subfile{chapters/characteristics/exercises}
    %%%%%%%%%%%%%%%%%%%%%%%%%%%%%%%%%%%%%%%%%%%%%%%%%%
    \chapter{ARIMA Models}
        \subfile{chapters/arima/backward_shift}
        \subfile{chapters/arima/smoothing}
        \subfile{chapters/arima/ar}
        \subfile{chapters/arima/ma}
        \subfile{chapters/arima/arma}
        \subfile{chapters/arima/arima}
        \subfile{chapters/arima/example}
    %%%%%%%%%%%%%%%%%%%%%%%%%%%%%%%%%%%%%%%%%%%%%%%%%%
\end{document}