\documentclass[../time_series_notes.tex]{subfiles}
\begin{document}
    %%%%%%%%%%%%%%%%%%%%%%%%%%%%%%%%%%%%%%%%%%%%%%%%%%%%%%%%%%%%
    \chapter{Characteristics of Time Series}
    The study of time series is devoted to application of statistical techniques to data collected over multiple time steps. Traditional techniques are not useful here due to strong correlation between successive observations.\newline
    Two separate but not necessarily mutually exclusive approaches exist to study time series
    \begin{enumerate}
        \item Time Domain approach
        This approach gives importance to lagged relationships, such as how what happens today will effect the obervation 7 days later.
        \item Frequency Domain approach
        This approach gives importance to investigation ofo cycles, such as economic cycle of periods of expansion/recession.
    \end{enumerate}

    A time series is a collection of random variables $x_{1}, x_{2}, x_{3}, \ldots$ and can in general be denoted by $\{x_{t}\}$. Such a collection of random variables indexed by time is also called a \textbf{random process} and the values collected are often called the \textbf{realization} of the random process. In the context of time series analysis, the two can be used interchangeably.\newline

    The sampling rate can serve an important purpose when representing time series data. An insufficient sampling rate can completely change the appearance of the data. For instance, when recording the motion of a wheel, it can appear moving backwards at a low enough recording rate. This distortion is called \textbf{aliasing}.

    %%%%%%%%%%%%%%%%%%%%%%%%%%%%%%
    \section{Time Series Formulations}
    %%%%%%%%%%%%%%%%%%%%
    \subsection{White Noise}
    The simplest time series is just a series of uncorrelated variables. It has three flavours
    \begin{enumerate}
        \item white noise $w_{t} \sim wn(0, \sigma_{w}^{2})$
        \item white independent noise $w_{t} \sim iid(0, \sigma_{w}^{2})$
        \item Gaussian white noise $w_{t} \sim iid \mathcal{N}(0, \sigma_{w}^{2})$
    \end{enumerate}
    Just the noise is particularly difficult to model and we introduce some kind of correlation and smoothness into the model.

    %%%%%%%%%%%%%%%%%%%%
    \subsection{Moving Average and Filtering}
    Any time series can be smoothed by considering moving averages. A very simple example can be taking the average of three consecutive observations
    \begin{align*}
        v_{t} = \frac{1}{3}(x_{t-1} + x_{t} + x_{t+1})
    \end{align*}
    Care must be taken to ensure that the sum of the coefficients of all the terms is one.\newline

    As is evident, if we take more number of such terms, the slower oscillations become more prominent in comparison to the faster oscillations making moving averages work as a kind of \textbf{filter}.

    %%%%%%%%%%%%%%%%%%%%
    \subsection{Autoregression}
    This involves using a regression like function to predict the value of the time series at the current timestep. A simple example would be
    \begin{align*}
        x_{t} = x_{t-1} + 0.9x_{t-2} + w_{t}
    \end{align*}
    where $w_{t}$ is noise. Here we are using the observations from the last two timesteps to make our predictions. This idea can be extended to include more terms and even difference of terms as part of the equation. We make an assumption for some initial values to calculate $x_{1}$ and $x_{2}$.

    %%%%%%%%%%%%%%%%%%%%
    \subsection{Random Walk with Drift}
    The prediction function can be defined as
    \begin{align*}
        x_{t} &= \delta + x_{t-1} + w_{t}\\
        x_{t} &= \delta t + \sum_{t} w_{t}
    \end{align*}
    where $\delta$ is the drift. Setting $\delta = 0$ is same as a random walk.

    %%%%%%%%%%%%%%%%%%%%
    \subsection{Signal in Noise}
    Sometimes, the signal can be seen as having a periodic component with some noise
    \begin{align*}
        x_{t} = A\cos (2\pi \omega t + \phi) + w_{t}
    \end{align*}
    where $\phi$ is the phase or the offset of the signal, $\omega$ is the time period and $A$ is the amplitude. $w_{t}$ is noise that we cannot predict.\newline

    Let $\sigma_{w}$ denote the variance of the noise. Then, \textbf{signal-to-noise ratio} is the ratio of the amplitude of the signal to the variance of noise. The higher this quantity, the easier it is to detect the signal. In real data, the signal is always obscured by some noise.


    %%%%%%%%%%%%%%%%%%%%%%%%%%%%%%
    \subsection{Measures of Dependence}
    A complete description of a time series where we observe $n$ variables at the time instances $t_{1}, t_{2}, \ldots, t_{n}$ can be given by the joint cumulative distribution function (with constants $c_{1}, c_{2}, \ldots, c_{n}$)
    \begin{align*}
        F(x_{1}, x_{2}, \ldots, x_{n}) = P(x_{1} \leq c_{1}, x_{2} \leq c_{2}, \ldots x_{n} \leq c_{n})
    \end{align*}
    But this is rarely easy to calculate and takes a tractable form when the variables are jointly normal. The marginal distributions in general are
    \begin{align*}
        F_{t}(x) &= P(x_{t} \leq x)\\
        f_{t}(x) &= \frac{\partial F_{t}(x)}{\partial x}
    \end{align*}

    
    %%%%%%%%%%%%%%%%%%%%
    \subsection{Mean Function} \label{mean_fun}
    The mean function is similar to the usual definition of the expected value
    \begin{align*}
        \mu_{xt} = E[x_{t}] =  \int_{-\inf}^{\inf} x f_{t}(x) dx
    \end{align*}

    \textbf{Smoothing a series does not change it's mean}.
    \begin{align*}
        v_{t} &= \frac{1}{3}(x_{t-1} + x_{t} + x_{t+1})\\
        E[v_{t}] &= \frac{1}{3}(3E[x_{t}]) = E[x_{t}]
    \end{align*}

    Consider the random walk defined earlier with noise $w_{t}$ and also consider it's mean
    \begin{align*}
        \mu_{t} = \delta t + \sum_{t} w_{t}\\
        E[\mu_{t}] = \delta t
    \end{align*}
    since the mean of noise is zero. Thus, at any given instance, the walk is expected to lie on the straight line $\delta t$ with some noise added on top of it.\newline
    In a similar manner, laying a noise on top of the signal will not change the mean value because of the additive nature. The mean of the noise will become zero and the mean of the new signal will be same as the mean of the original signal.

    
    %%%%%%%%%%%%%%%%%%%%
    \subsection{Autocovariance Function}
    For any two time points $s$ and $t$,
    \begin{align*}
        \gamma_{x}(s, t) = E[(x_{s} - \mu_{s})(x_{t} - \mu_{t})]
    \end{align*}
    which is similar to the covariance between the two points. Note that the function is symmetric and $\gamma_{x}(s, t) = \gamma_{x}(t, s)$.\newline

    The autocovariance is a measure of linear dependence between two points of the same time series. \textbf{For very smooth series, the autcovariance tends to take large values while for choppy series, it tends to take very small values}.\newline

    Consider the example of a white noise series. We know that any nearby points are not correlated with one another. The same will hold with the autocovariance values as well. It will be zero when $s \neq t$ and $\sigma_{w}^{2}$ when $s = t$.\newline

    \subsection{Covariance of Linear Combinations}
    For linear combinations of random variables $X_{i}$ and $Y_{j}$, the following holds
    \begin{align*}
        U &= \sum_{i=1}^{n} a_{i} X_{i}\\
        V &= \sum_{j=1}^{m} b_{j} Y_{j}\\
        Cov(U,V) &= \sum_{i=1}^{n} \sum_{j=1}^{m} a_{i} b_{j} Cov(X_{i}, Y_{j})\\
        Var(U) &= Cov(U,U)
    \end{align*}
    As an example, consider the covariance calculation for the moving averages of white noise defined as
    \begin{align*}
        v_{t} &= \frac{1}{3}(w_{t-1} + w_{t} + w_{t+1})\\
        Cov(t, t) &= \frac{1}{9}(Cov(w_{t-1}, w_{t-1}) + Cov(w_{t}, w_{t}) + Cov(w_{t+1}, w_{t+1}))\\
        &= \frac{3}{9} \sigma_{w}^{2}
    \end{align*}
    because the covariance between any two different points in the white noise time series is zero and the covariance of a point with itself will be the variance.\newline
    Continuing the calculation for different time points
    \begin{align*}
        Cov(t+1, t) &= Cov(\frac{1}{3}(w_{t} + w_{t+1} + w_{t+2}), \frac{1}{3}(w_{t-1} + w_{t} + w_{t+1}))\\
        &= \frac{1}{9}(Cov(w_{t}, w_{t}) + Cov(w_{t+1}, w_{t+1}))\\
        &= \frac{2}{9}\sigma_{w}^{2}
    \end{align*}
    If we continue the calculation, we will observe that the covariance diminishes once we calculate the value between two time points which are more than three time steps apart. This should be the case since then the two points for which the covariance is being calculated will not have any overlapping terms.
    \begin{align*}
        \hypertarget{ma_noise_cov}{\gamma(s,t)} = \begin{cases} \frac{3}{9}\sigma_{w}^{2} &\mbox{$s = t$},\\
                                    \frac{2}{9}\sigma_{w}^{2} &\mbox{$\lvert s - t \rvert = 1$}\\
                                    \frac{1}{9}\sigma_{w}^{2} &\mbox{$\lvert s - t \rvert = 2$}\\
                                    0 &\mbox{otherwise} \end{cases}
    \end{align*}


    %%%%%%%%%%%%%%%%%%%%
    \subsection{Autocorrelation Function (ACF)}
    This measure tries to bring the auto correlation value in the range of $[-1,1]$
    \begin{align*}
        \rho(s,t) = \frac{\gamma_{x}(s,t)}{\sqrt{\gamma_{x}(s,s) \gamma_{x}(t,t)}}
    \end{align*}
    With the \emph{Cauchy Schwarz Inequality}, the numerator is always less than or equal to the denominator, bringing the value in the range $[-1,1]$. This value also gives a rough measure of the ability to forecast the time series (correlation 1 implies perfect relationship and thus we can forecast with zero error).

    %%%%%%%%%%%%%%%%%%%%
    \subsection{Cross Covariance and Cross Correlation Function (CCF)}
    The above defined formulae applied to two different points in the same time series. The definitions extend to different time series as well
    \begin{align*}
        \gamma_{xy}(s,t) = E[(x_{s} - \mu_{xs})(x_{t} - \mu_{yt})]\\
        \rho_{xy}(s,t) = \frac{\gamma_{xy}(s,t)}{\sqrt{\gamma_{x}(s,s) \gamma_{y}(t,t)}}
    \end{align*}
    The definitions extend well to multivariate case as well, where we define the above values for pairs of different time series.

    
    %%%%%%%%%%%%%%%%%%%%%%%%%%%%%%
    \subsection{Stationary Time Series}
    A \textbf{strictly stationary time series is the one for which the probabilistic behaviour is invariant to time shifts}. Mathematically,
    \begin{gather*}
        P(x_{t_{1}} \leq c_{1}, x_{t_{2}} \leq c_{2}, \ldots, x_{t_{n}} \leq c_{n}) = P(x_{t_{1}+h} \leq c_{1}, x_{t_{2}+h} \leq c_{2}, \ldots, x_{t_{n}+h} \leq c_{n})\\
    \forall k = 0,1,2,\ldots, \forall t_{1}, t_{2}, \ldots, t_{n} \text{ and } \forall h = 0, \pm 1, \pm 2, \ldots
    \end{gather*}

    For $k = 1$, the equation implies
    \begin{align*}
        P(x_{t} \leq c) = P(x_{s} \leq c)
    \end{align*}
    and by differentiating this, we will get the density function which is same for both time points. Thus, the mean of the series will be same, i.e. $\mu_{t} = \mu_{s}$, and this remains true for all the time points implying, \textbf{for a strictly stationary time series, mean $\mu$ is constant throughout}.\newline

    For $k=2$, the equation implies
    \begin{align*}
        P(x_{t} \leq c_{1}, x_{s} \leq c_{2}) = P(x_{t+h} \leq c_{1}, x_{s+h} \leq c_{2})
    \end{align*}
    and by differentiating this, we will get the joint density of $x_{t}, x_{s}$ which is same even after shifting both the quantities by $h$. Note that the autcovariance $\gamma_{st} = E[x_{t} x_{s}] - \mu_{t} \mu_{s}$. From $k=1$, we have established that the mean remains constant throughout. Further, we just showed that the joint density is identical, meaning the expected value will also be identical. Hence, \textbf{for strictly stationary time series, the autocovariance is dependent only on the difference between the two time steps and not the location}.\newline

    \paragraph{Wealy Stationary Time Series} satisfies the following properties
    \begin{enumerate}
        \item The mean $\mu$ is constant and not dependent on time.
        \item The autocovariance $\gamma_{st}$ depends only on $\lvert s - t \rvert$ and not on the location of the time steps.
    \end{enumerate}
    Thus, we remove any claims for same probability distributions for values of $k > 2$.\newline

    \subsection{Autocovariance and Autocorrelation Function}
    The mean of a stationary series is constant. Let it be denoted by $\mu$.\newline
    The autocovariance is only dependent on the difference between the time steps
    \begin{align*}
        \gamma (h) = \gamma(x_{t}, x_{t+h}) = E[(x_{t} - \mu)(x_{t+h} - \mu)]
    \end{align*}

    The autocorrelation can then be simply defined as
    \begin{align*}
        \rho (h) = \rho(x_{t}, x_{t+h}) = \frac{\gamma (t,t+h)}{\sqrt{\gamma(t, t), \gamma(t+h, t+h)}} = \frac{\gamma(h)}{\gamma(0)}
    \end{align*}
    The denominator could be simplified because of the fact that the autocovariance is only dependent on the time difference and not the location of the time.\newline

    If we check at the stationarity of the moving average of white noise, we quickly notice that the mean is constant (from \ref{mean_fun}) and the autocovariance is only dependent on the difference of the two time steps (\hyperlink{ma_noise_cov}{from}). Thus the series is stationary (weakly).

    \subsection{Joint Stationarity}
    \textbf{Two stationary time series are said to be jointly stationary if the autocovariance of the two series is only dependent on the lag $h$}.\newline
    \begin{align*}
        \gamma_{xy}(h) = Cov(x_{t+h}, y_{t}) = E[(x_{t+h} - \mu_{x})(y_{t} - \mu_{y})]
    \end{align*}
    is only dependent on the lag $h$ and not on the specific time $t$.\newline

    The \textbf{Cross Correlation Function} is then equivalent to
    \begin{align*}
        \rho_{xy}(h) = \frac{\gamma_{xy}(h)}{\sqrt{\gamma_{x}(0) \gamma_{y}(0)}}
    \end{align*}
    which will lie in the range $[-1,1]$.

\end{document}